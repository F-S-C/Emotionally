% Copyright (c)  2019  FSC.
% Permission is granted to copy, distribute and/or modify this document
% under the terms of the GNU Free Documentation License, Version 1.3
% or any later version published by the Free Software Foundation;
% with no Invariant Sections, no Front-Cover Texts, and no Back-Cover Texts.
% A copy of the license is included in the section entitled "GNU
% Free Documentation License".

\chapter{Test}\label{chap:test}
Nella seguente sezione ci si concentra maggiormente sula fase di test del 
prodotto, analizzandone alcuni scenari di prova e integrandone le modifiche 
apportare in fase di alpha test e beta test.

\section{Scenari e casi di prova}\label{sec:scenari-casi-prova}
\paragraph{Scenari di prova per il caso d'uso 'Create project'}
Scenari di successo:
\begin{itemize}
	\item Inserimento progetto
\end{itemize}
Scenari di insuccesso:
\begin{itemize}
	\item Inserimento fallito a causa del nome scelto per il progetto già 
	esistente
\end{itemize}

\paragraph{Scenari di prova per il caso d'uso 'Update user data'}
Scenari di successo: 
\begin{itemize}
	\item Modificato il profilo
	\begin{itemize}
		\item Modificato il nome
		\item Modificato il cognome
		\item Modificata la password
	\end{itemize}
\end{itemize}
Scenari di insuccesso:
\begin{itemize}
	\item Modifica fallita a causa della vecchia password errata
	\item Modifica fallita a causa della non uguaglianza tra la nuova password 
	e conferma password (solo in caso si voglia modificare la password)
\end{itemize}

\section{Alpha test}\label{sec:alpha-test}
Non è stato condotto un vero e proprio alpha test, ma seguendo il modello 
basato sui prototipi successivi si sono effettuati i test in ogni fase della 
realizzazione di ogni singolo prototipo.

Per quanto riguarda la realizzazione del sistema, ad ogni item realizzato si 
sono effettuati i test di prova per convalidare la funzionalità testata 
evitando di portarsi uno o più errori fase della realizzazione.

\section{Beta test}
I seguenti dati riportati nella seguente tabella sono i problemi riscontrati 
dai soggetti intervistati (con dati anonimi) durante la fase di Beta Test, 
cercando di sollecitare il sistema nel maggior numero di modi possibili e 
verificando che esso si comporti secondo le aspettative.

\begin{table}[]
	\begin{tabular}{ccccccc}
		\textbf{Problema\#} & \textbf{Rilevato da} & \textbf{in data} & 
		\textbf{Caso d'uso} & \textbf{Descrizione del 
		problema}                                                                                                        &
		 \textbf{Gravità} & \textbf{Risolto in data} \\
		1                   & Soggetto 1           & 20/02/202        & Upload 
		video        & \begin{tabular}[c]{@{}c@{}}Il sistema, se avviato con 
		Google \\ Chrome, non permette l'inserimento \\ di video in real 
		time\end{tabular} & 3                & -                        \\
		1                   & Soggetto 2           & 21/02/2020       & Upload 
		video        & \begin{tabular}[c]{@{}c@{}}Il sistema, se avviato con 
		Google \\ Chrome, non permette l'inserimento \\ di video in real 
		time\end{tabular} & 3                & -                        \\
		4                   & Soggetto 4           & 21/02/2020       & View 
		report         & \begin{tabular}[c]{@{}c@{}}Ogni tanto, anche se non ci 
		sono \\ video, nel report si creano grafici\\ valorizzati a 
		0\end{tabular}        & 1                & -                       
	\end{tabular}
\end{table}

Gravità:
	\begin{enumerate}
		\item problema molto lieve; può essere risolto dopo la pubblicazione 
		del sito
		\item problema lieve, da risolvere prima della pubblicazione del sito
		\item problema grave ma bypassabile; non pregiudica l'uso del sito
		\item problema bloccante, impedisce l'uso della funzione
	\end{enumerate}

\paragraph{Note} Il problema \textbf{#1} è dovuto a un bug presente nel browser 
Google Chrome che non da la possibilità di poter gestire video real time. Come 
già riportato nei requisiti di usabilità, per il funzionamento corretto del 
sistema, è consigliabile l'utilizzo di FireFox.