% Copyright (c)  2019  FSC.
% Permission is granted to copy, distribute and/or modify this document
% under the terms of the GNU Free Documentation License, Version 1.3
% or any later version published by the Free Software Foundation;
% with no Invariant Sections, no Front-Cover Texts, and no Back-Cover Texts.
% A copy of the license is included in the section entitled "GNU
% Free Documentation License".

\chapter{Piano di qualità}\label{chap:piano-qualita}

\section{Analisi dei rischi}\label{sec:analisi-rischi}
%TODO GDPR

\section{Piano del progetto}\label{sec:piano-progetto}
%TODO Gantt diagram

\section{Organizzazione del gruppo}\label{sec:organizzazione-gruppo}
Il gruppo, dopo essersi riunito in un brainsorming iniziale, ha deciso di 
dividersi in maniera modulare assegnando ad ogni componente un compito 
specifico.

In ogni fase, tutti i componenti del gruppo si riuniscono per una 
pianificazione iniziale e, alla fine di essa, si occupano  di verificare e 
convalidare il lavoro svolto.

Di seguito viene riportata la suddivisione dei compiti:

\begin{itemize}
	\item Fase 1
	\item Fase 2
	\item Fase 3
	\item Fase 4
	\item Fase 5
	\item Fase 6
\end{itemize}

