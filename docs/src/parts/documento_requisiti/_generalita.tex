% Copyright (c)  2019  FSC.
% Permission is granted to copy, distribute and/or modify this document
% under the terms of the GNU Free Documentation License, Version 1.3
% or any later version published by the Free Software Foundation;
% with no Invariant Sections, no Front-Cover Texts, and no Back-Cover Texts.
% A copy of the license is included in the section entitled "GNU
% Free Documentation License".

\chapter{Generalità}\label{chap:generalita}

Il sistema ``Emotionally'' ha l'obiettivo di semplificare la \textit{sentiment
analysis} agli analisti di tutti i campi. Il nome del sistema deriva dalla
contrazione delle parole inglesi ``\textsc{emotional} ana\textsc{ly}sis''.

\section{Il committente}\label{sec:il-committente}

L'analisi delle emozioni è, a oggi, una componente importante per lo studio
dell'impatto di prodotti e servizi sugli utenti finali. Avere a disposizione un
\textit{tool} che ne permette la semi-automatizzazione è senz'altro di
un'utilità indiscutibile.

Il committente di questo progetto è il professor Giuseppe Desolda
dell'Università degli Studi di Bari, che ha come obiettivo quello di realizzare
un sistema in grado di analizzare le emozioni degli utenti analizzando dei
video. La richiesta del committente è quella di utilizzare le API del sistema
Affectiva, uno dei migliori tool di questa tipologia.

\section{Situazione attuale}\label{sec:situazione-attuale}

Al momento non si ha una base su cui poter effettuare delle analisi. Per questo,
seguendo il flusso di sviluppo scelto, si procederà all'analisi della
concorrenza. 

\section{Obiettivi generali del nuovo sito}\label{sec:obiettivi-generali-del-nuovo-sito}
\subsection{Obiettivi generali}\label{subsec:obiettivi-generali}

Ad oggi sono pochi i sistemi che permettono di esaminare le emozioni attraverso
flussi di video raccogliendo le analisi in formato statistico. I dati raccolti
possono essere utili per analizzare campi applicativi più o meno vasti (ad
esempio Marketing, Produzione Multimediale, Usabilità, Accessibilità
\textit{etc...}). 

\subsection{Obiettivi specifici}\label{subsec:obiettivi-specifici}

Gli obiettivi che il team si è posto di raggiungere per la prima release del
sistema sono:

\begin{itemize}
	\item L'analisi delle emozioni provate dai soggetti di un video
	\item Riportare le analisi effettuate in forma più o meno dettagliata
	\item Creare e gestire progetti con uno o più video
	\item Permettere di aggregare i dati di due o più video di uno stesso
	progetto
	\item Permettere la condivisione di un progetto con più utenti
\end{itemize}

\section{Gli utenti}\label{sec:gli-utenti}

Gli Utenti utilizzatori della piattaforma si dividono principalmente in due
categorie: \texttt{Analista} e \texttt{Designer}. Le relazioni che intercorrono
fra le due categorie, sono rappresentate in \autoref{fig:tipi-utente}.

\begin{figure}[H]
	\centering
	\begin{tikzpicture}[node distance=2cm]
		\umlsimpleclass[type=abstract, x=0, y=0]{Utente}
		\umlsimpleclass[x=-3, y=-3]{Analista}
		\umlsimpleclass[x=3, y=-3]{Designer}
		\umlinherit[geometry=|-|]{Analista}{Utente}
		\umlinherit[geometry=|-|]{Designer}{Utente}
	\end{tikzpicture}
	\caption{Relazione tra le tipologie di utente di Emotionally}
	\label{fig:tipi-utente}
\end{figure}

\section{Scenari d'uso}\label{sec:scenari-duso}

\section{Posizionamento competitivo}\label{sec:posizionamento-competitivo}
