% Copyright (c)  2019  FSC.
% Permission is granted to copy, distribute and/or modify this document
% under the terms of the GNU Free Documentation License, Version 1.3
% or any later version published by the Free Software Foundation;
% with no Invariant Sections, no Front-Cover Texts, and no Back-Cover Texts.
% A copy of the license is included in the section entitled "GNU
% Free Documentation License".

\chapter{Generalità}\label{chap:generalita}

Il sistema ``Emotionally'' ha l'obiettivo di semplificare la \textit{sentiment
	analysis} agli analisti di tutti i campi. Il nome del sistema deriva dalla
contrazione delle parole inglesi ``\textsc{emotional} ana\textsc{ly}sis''.

\section{Il committente}\label{sec:il-committente}

L'analisi delle emozioni è, a oggi, una componente importante per lo studio
dell'impatto di prodotti e servizi sugli utenti finali. Avere a disposizione un
\textit{tool} che ne permette la semi-automatizzazione è senz'altro di
un'utilità indiscutibile.

Il committente di questo progetto è il Professor Giuseppe Desolda
dell'Università degli Studi di Bari, che ha come obiettivo quello di realizzare
un sistema in grado di analizzare le emozioni degli utenti analizzando dei
video. La richiesta del committente è quella di utilizzare le API del sistema
Affectiva, uno dei migliori tool di questa tipologia.

\section{Situazione attuale}\label{sec:situazione-attuale}

Al momento non si ha una base su cui poter effettuare delle analisi. Per questo,
seguendo il flusso di sviluppo scelto, si procederà all'analisi della
concorrenza.

\section{Obiettivi generali del nuovo sito}\label{sec:obiettivi-generali-del-nuovo-sito}
\subsection{Obiettivi generali}\label{subsec:obiettivi-generali}

Ad oggi sono pochi i sistemi che permettono di esaminare le emozioni attraverso
flussi di video raccogliendo le analisi in formato statistico. I dati raccolti
possono essere utili in vari campi applicativi (ad esempio Marketing, Produzione
Multimediale, Usabilità, Accessibilità \textit{etc...}), in misura più o meno
vasta.

\subsection{Obiettivi specifici}\label{subsec:obiettivi-specifici}

Gli obiettivi che il team si è posto di raggiungere per la prima
\textit{release} del sistema sono:

\begin{itemize}
	\item L'analisi delle emozioni provate dai soggetti di un video
	\item Riportare le analisi effettuate in forma più o meno dettagliata
	\item Creare e gestire progetti con uno o più video
	\item Permettere di aggregare i dati di due o più video di uno stesso
	      progetto
	\item Permettere la condivisione di un progetto con più utenti
\end{itemize}

\section{Gli utenti}\label{sec:gli-utenti}

Gli Utenti utilizzatori della piattaforma si dividono principalmente in due
categorie: \textbf{Analista} e \textbf{Designer}. Le relazioni che intercorrono
fra le due categorie, sono rappresentate in \autoref{fig:tipi-utente}.

\begin{table}[H]
	\centering
	\caption{I bisogni degli utenti di Emotionally.}
	\label{tab:bisogni-utenti}
	\rowcolors{2}{gray!25}{white!0}
	\begin{longtable}{@{}|>{\centering\arraybackslash}m{.25\textwidth}|m{.5\textwidth}|>{\centering\arraybackslash}m{.1\textwidth}|@{}}
		\hline
		\rowcolor{emotionally-color}
		{\color{white} \textbf{Categoria di utente}}   & {\color{white} \textbf{Bisogni principali in relazione al sito}}     & {\color{white} \textbf{Priorità}} \\\hline
		\endfirsthead
		\cellcolor{white!0}                            & Analizzare video o gruppi di video                                   & Alta                              \\
		\cellcolor{white!0}                            & Visualizzare dei report delle analisi effettuate                     & Alta                              \\
		\cellcolor{white!0}                            & Esportare i report delle analisi effettuate                          & Media                             \\
		\cellcolor{white!0}                            & Condividere i progetti creati con altri utenti                       & Media                             \\
		\multirow{-5}{*}{Analista}                     & Analizzare dei video registrati in tempo reale                       & Alta                              \\
		\cellcolor{gray!25}                            & Visualizzare i report di analisi effettuate da altri utenti          & Alta                              \\
		\multirow{-2}{*}{\cellcolor{gray!25} Designer} & Visualizzazione dei dati in formati strutturati e formali (es. JSON) & Alta                              \\
		\hline
	\end{longtable}
\end{table}
\begin{table}[H]
	\centering
	\caption{Gli obiettivi del commitente legati agli utenti di Emotionally.}
	\label{tab:obiettivi-committente-utenti}
	\rowcolors{2}{gray!25}{white!0}
	\begin{longtable}{@{}|>{\centering\arraybackslash}m{.25\textwidth}|m{.5\textwidth}|>{\centering\arraybackslash}m{.1\textwidth}|@{}}
		\hline
		\rowcolor{emotionally-color}
		{\color{white} \textbf{Categoria di utente}} & {\color{white} \textbf{Obiettivi del commitente}} & {\color{white} \textbf{Priorità}} \\\hline
		\endfirsthead
		\cellcolor{white!0}                          & Rendere efficiente l'analisi di 
		singoli video o gruppi di video                & 
		Alta                              \\
		\cellcolor{white!0}                          & Generare automaticamente i report 
		delle analisi effettuate  & Alta                              \\
		\multirow{-3}{*}{Analista}                   & Migliorare la comunicazione 
		attraverso la condivisione di progetto    & Media                              \\
		\cellcolor{gray!25}                            & Generare automaticamente i 
		report in formati strutturati e formali (es. JSON)          & 
		Alta                              \\
		\multirow{-2}{*}{\cellcolor{gray!25} Designer} & Migliorare la comunicazione 
		attraverso la condivisione di progetto & Media                              \\
		\hline
	\end{longtable}
\end{table}
\begin{figure}[H]
	\centering
	\begin{tikzpicture}[node distance=2cm]
		\umlsimpleclass[type=abstract, x=0, y=0]{Utente}
		\umlsimpleclass[x=-1.5, y=-2]{Analista}
		\umlsimpleclass[x=1.5, y=-2]{Designer}
		\umlinherit[geometry=|-|]{Analista}{Utente}
		\umlinherit[geometry=|-|]{Designer}{Utente}
	\end{tikzpicture}
	\caption{Relazione tra le tipologie di utente di Emotionally}
	\label{fig:tipi-utente}
\end{figure}


\section{Scenari d'uso}\label{sec:scenari-duso}
Dei possibili scenari d'uso possono essere:
\begin{itemize}
	\item Un analista utilizza questo sistema per poter effettuare un'analisi 
	delle emozioni degli utenti che visualizzano una pubblicità, un sito web o 
	quant'altro e ricavare dei risultati da utilizzare per poter capire ciò che 
	piace all'utente, quali emozioni hanno provato, quale riscontro ha 
	l'elemento in analisi nei confronti dell'utente e così via.
	\item Un designer utilizza questo sistema per poter accedere ai report 
	precedentemente generati (solitamente da un analista) e scaricarne i 
	risultati in vari formati compatibili con strumenti di design e 
	programmazione per poterli utilizzare nei propri progetti.
\end{itemize}

\section{Posizionamento competitivo}\label{sec:posizionamento-competitivo}

Il posizionamento competitivo di Emotionally è di offrire una piattaforma Open 
Source tentando di contrastare il ``monopolio'' di tool a pagamento offrendo 
una piattaforma utilizzabile gratuitamente. Uno dei bonus rispetto ai 
competitor è la natura web del sistema che non richiede alcun tipo di 
settaggio. Inoltre, si sfruttano le API di Affectiva, uno dei migliori 
\textit{engine} di \textit{sentiment analysis}.
