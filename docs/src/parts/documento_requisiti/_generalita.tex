% Copyright (c)  2019  FSC.
% Permission is granted to copy, distribute and/or modify this document
% under the terms of the GNU Free Documentation License, Version 1.3
% or any later version published by the Free Software Foundation;
% with no Invariant Sections, no Front-Cover Texts, and no Back-Cover Texts.
% A copy of the license is included in the section entitled "GNU
% Free Documentation License".

\chapter{Generalità}\label{chap:generalita}

Il sistema ``Emotionally'' ha l'obiettivo di semplificare la \textit{sentiment
analysis} agli analisti di tutti i campi. Il nome del sistema deriva dalla
contrazione delle parole inglesi ``\textsc{emotional} ana\textsc{ly}sis''.

\section{Il committente}\label{sec:il-committente}

L'analisi delle emozioni è, a oggi, una componente importante per lo studio
dell'impatto di prodotti e servizi sugli utenti finali. Avere a disposizione un
\textit{tool} che ne permette la semi-automatizzazione è senz'altro di
un'utilità indiscutibile.

Il committente di questo progetto è il professor Giuseppe Desolda
dell'Università degli Studi di Bari, che ha come obiettivo quello di realizzare
un sistema in grado di analizzare le emozioni degli utenti analizzando dei
video. La richiesta del committente è quella di utilizzare le API del sistema
Affectiva, uno dei migliori tool di questa tipologia.

\section{Situazione attuale}\label{sec:situazione-attuale}

\section{Obiettivi generali del nuovo sito}\label{sec:obiettivi-generali-del-nuovo-sito}

\section{Gli utenti}\label{sec:gli-utenti}

\section{Scenari d'uso}\label{sec:scenari-duso}

\section{Posizionamento competitivo}\label{sec:posizionamento-competitivo}
