% Copyright (c)  2019  FSC.
% Permission is granted to copy, distribute and/or modify this document
% under the terms of the GNU Free Documentation License, Version 1.3
% or any later version published by the Free Software Foundation;
% with no Invariant Sections, no Front-Cover Texts, and no Back-Cover Texts.
% A copy of the license is included in the section entitled "GNU
% Free Documentation License".

\chapter{Requisiti del sito}\label{chap:requisiti-del-sito}

\section{Requisiti di architettura}\label{sec:requisiti-di-architettura}

L'architettura informativa illustrata successivamente definisce una struttura 
generale dello \textit{schema gerarchico} indicante la logica di 
\textit{navigazione} suddividendo la web app in sezioni e sottosezioni. 

Per quanto riguarda l'area pubblica si prevede quanto segue:
\begin{itemize}
	\item \textbf{Landing page}
	\begin{itemize}
		\item \textbf{Funzionalità}
		\item \textbf{Su di noi}
	\end{itemize}
\item \textbf{Login}
\item \textbf{Registrazione}
\end{itemize}

Per quanto riguarda l'area privata, ovvero per ogni utente loggato si prevede 
quanto segue (\textit{con * viene indita la molteplice quantità dell'oggetto a 
cui è affiancato}): 
\begin{itemize}
	\item \textbf{Home}
	\begin{itemize}
		\item \textbf{Progetti *} 
		\begin{itemize}
			\item \textbf{Sottoprogetti *}
			\begin{itemize}
				\item \textbf{Video *}
			\end{itemize}
		\item \textbf{Video *}
		\end{itemize}
	\end{itemize}
\end{itemize}

Indicativamente, la struttura di navigazione delle pagine più importante è 
definita mediante le seguenti \textit{gabbie logiche grafiche minimali}. Si 
riserverà al web designer il compito di definire meccanismi di navigazione e 
layout grafici più avanzati che ne determineranno il risultato finale.

% TODO: Insert a logical cages.

\section{Requisiti di comunicazione}\label{sec:requisiti-di-comunicazione}
Le pagine interne della piattaforma web Emotionally dovranno essere 
graficamente coerenti tra di loro, fatta eccezione per la pagina di landing che 
dovrà presentare la piattaforma all'utente. La piattaforma web sarà bilingue 
(italiano e inglese).

Ogni pagina del sistema Emotionally dovrà:
\begin{itemize}
	\item contenere il logo del sistema stesso posizionato in alto a sinistra,
	\item avere un comportamento responsive: il sito deve ridimensionarsi 
	automaticamente al variare della risoluzionie video e della dimnesione 
	della finestra del browser,
	\item contenere caratteri la cui dimensione dovrà essere modificabile 
	dall'utente,
	\item avere una funzione accessibile, almeno di livello AA.
\end{itemize}
La landing page presenta, attraverso un'interfaccia chiara e usabile, le 
funzionalità del sistema e permette all'utente di approcciare il riconoscimento 
delle emozioni. 

Inoltre, per sottolineare la divisione tra il sistema e la landing page si è 
deciso di utilizzare un bottone di login differente rispetto allo stile dei 
link presenti nella navbar. A seguito del click del bottone, l'utente viene 
reindirizzato alla pagina di login che rappresenta, seppur in parte, lo stile 
grafico del sistema.

\section{Requisiti funzionali}\label{sec:requisiti-funzionali}

In questa sezione vengono illustrati i casi d'uso e le relative 
rappresentazioni tabellari riferiti alle diverse funzionalità che Emotionally 
offre ai suoi utenti. 

\subsection{Casi d'uso}

\begin{figure}[H]
	\centering
    \caption{I casi d'uso del sistema.}
    \label{fig:casi-duso}
    \resizebox{\textwidth}{!}{%
        % Copyright (c)  2019  FSC.
% Permission is granted to copy, distribute and/or modify this document
% under the terms of the GNU Free Documentation License, Version 1.3
% or any later version published by the Free Software Foundation;
% with no Invariant Sections, no Front-Cover Texts, and no Back-Cover Texts.
% A copy of the license is included in the section entitled "GNU
% Free Documentation License".

\begin{tikzpicture} 
    \begin{umlsystem}[x=6, y=20]{Emotionally} 
        % Define uses case of the GUEST
        \umlusecase[x=0,y=-2, name=Registration]{Registration} 
        \umlusecase[x=0,y=-3, name=TestAnalysis]{Test analysis}
        
        % Define uses case of the USER
        \umlusecase[x=0,y=-4, name=Login]{Login}
        \umlusecase[x=0,y=-5, name=Logout]{Logout}
        
        \umlusecase[x=0,y=-6, name=ProjectCreation]{Create project}
        \umlusecase[x=8, y=-2, name=ProjectEditing]{Edit project}
        \umlusecase[x=8, y=-3, name=ProjectRemoval]{Remove project}
        \umlusecase[x=0,y=-7, name=ProjectView]{View project}
        
        \umlusecase[x=8, y=-4, name=VideoUpload]{Upload video}
        \umlusecase[x=8, y=-5, name=VideoEditing]{Edit video}
        \umlusecase[x=8, y=-6, name=VideoRemoval]{Remove video}
        \umlusecase[x=8, y=-7, name=ViewVideo]{View video}
        
        \umlusecase[x=8, y=-8, name=SharingProject]{Share project}
        
        \umlusecase[y=-10, name=ReportView]{View report}
        \umlusecase[x=8, y=-9.5, name=VideoReportView]{View report of a video}
        \umlusecase[x=8, y=-10.5, name=ProjectReportView]{View report of a project}
        
        \umlusecase[x=8, y=-11.5, name=ReportDownload]{Download report}
        
        \umlusecase[y=-9, name=UserDataUpdate]{Update user data}
        \umlusecase[y=-8, name=DeleteAccount]{Delete account}
    \end{umlsystem} 
    
    % Define actor
    \umlactor[x=1, y=17.5]{Guest}
    \umlactor[x=1, y=14.5]{User}
    \umlactor[y=10.5]{Designer}
    \umlactor[x=2, y=10.5]{Analyst}
    
    % Define generalizzation between actor
    \umlinherit{User}{Guest}
    \umlinherit[geometry=|-|]{Designer}{User}
    \umlinherit[geometry=|-|]{Analyst}{User}
    
    % Define associations between actors and use cases
    \umlassoc{Guest}{Registration} 
    \umlassoc{Guest}{TestAnalysis} 
    \umlassoc{User}{Login} 
    \umlassoc{User}{Logout} 
    \umlassoc{User}{ProjectCreation}
    \umlassoc{User}{ProjectView}
    \umlassoc{User}{ReportView}
    \umlassoc{User}{UserDataUpdate}
    \umlassoc{User}{DeleteAccount}
    
    %Define extensions between use cases
    \umlHVHextend{ProjectEditing}{ProjectView}
    \umlHVHextend{ProjectRemoval}{ProjectView}
    \umlHVHextend{SharingProject}{ProjectView}
    \umlHVHextend{VideoUpload}{ProjectView}
    \umlHVHextend{VideoEditing}{ProjectView}
    \umlHVHextend{VideoRemoval}{ProjectView}
    \umlHVHextend{ViewVideo}{ProjectView}
    \umlHVextend{ReportDownload}{ReportView}
    
    % Define generalizzation between use cases
    \umlinherit[geometry=-|-]{VideoReportView}{ReportView}
    \umlinherit[geometry=-|-]{ProjectReportView}{ReportView}
\end{tikzpicture}

    }
\end{figure}

\subsection{Rappresentazione tabellare}
% TODO: Complete use case scenarios
\begin{table}[H]
	\centering
	\caption{Use Case: Test analysis}
	\label{tab:use-case-test-analysis}
	\rowcolors{2}{gray!25}{white!0}
	\begin{longtable}{@{}|>{\centering\arraybackslash}m{.25\textwidth}|m{.5\textwidth}|>{\centering\arraybackslash}m{.1\textwidth}|@{}}
		
		Nome caso d'uso & Registration \\
		Descrizione & \\
		Attori & \\
		Pre-condizioni & \\
		Sequenza delle azioni & \\
		Post-condizioni & \\
		Scenario alternativo & \\
		
	\end{longtable}
\end{table}

\begin{table}[H]
	\centering
	\caption{Use Case: Registration}
	\label{tab:use-case-registration}
	\rowcolors{2}{gray!25}{white!0}
	\begin{longtable}{@{}|>{\centering\arraybackslash}m{.25\textwidth}|m{.5\textwidth}|>{\centering\arraybackslash}m{.1\textwidth}|@{}}
		
		Nome caso d'uso & Test analysis \\
		Descrizione & \\
		Attori & \\
		Pre-condizioni & \\
		Sequenza delle azioni & \\
		Post-condizioni & \\
		Scenario alternativo & \\
		
	\end{longtable}
\end{table}

\begin{table}[H]
	\centering
	\caption{Use Case: Login}
	\label{tab:use-case-login}
	\rowcolors{2}{gray!25}{white!0}
	\begin{longtable}{@{}|>{\centering\arraybackslash}m{.25\textwidth}|m{.5\textwidth}|>{\centering\arraybackslash}m{.1\textwidth}|@{}}
		
		Nome caso d'uso & Login \\
		Descrizione & \\
		Attori & \\
		Pre-condizioni & \\
		Sequenza delle azioni & \\
		Post-condizioni & \\
		Scenario alternativo & \\
		
	\end{longtable}
\end{table}

\begin{table}[H]
	\centering
	\caption{Use Case: Logout}
	\label{tab:use-case-logout}
	\rowcolors{2}{gray!25}{white!0}
	\begin{longtable}{@{}|>{\centering\arraybackslash}m{.25\textwidth}|m{.5\textwidth}|>{\centering\arraybackslash}m{.1\textwidth}|@{}}
		
		Nome caso d'uso & Logout \\
		Descrizione & \\
		Attori & \\
		Pre-condizioni & \\
		Sequenza delle azioni & \\
		Post-condizioni & \\
		Scenario alternativo & \\
		
	\end{longtable}
\end{table}

\begin{table}[H]
	\centering
	\caption{Use Case: Create project}
	\label{tab:use-case-project-creation}
	\rowcolors{2}{gray!25}{white!0}
	\begin{longtable}{@{}|>{\centering\arraybackslash}m{.25\textwidth}|m{.5\textwidth}|>{\centering\arraybackslash}m{.1\textwidth}|@{}}
		
		Nome caso d'uso & Create project \\
		Descrizione & \\
		Attori & \\
		Pre-condizioni & \\
		Sequenza delle azioni & \\
		Post-condizioni & \\
		Scenario alternativo & \\
		
	\end{longtable}
\end{table}

\begin{table}[H]
	\centering
	\caption{Use Case: View project}
	\label{tab:use-case-project-view}
	\rowcolors{2}{gray!25}{white!0}
	\begin{longtable}{@{}|>{\centering\arraybackslash}m{.25\textwidth}|m{.5\textwidth}|>{\centering\arraybackslash}m{.1\textwidth}|@{}}
		
		Nome caso d'uso & View project \\
		Descrizione & \\
		Attori & \\
		Pre-condizioni & \\
		Sequenza delle azioni & \\
		Post-condizioni & \\
		Scenario alternativo & \\
		
	\end{longtable}
\end{table}

\begin{table}[H]
	\centering
	\caption{Use Case: Edit project}
	\label{tab:use-case-project-editing}
	\rowcolors{2}{gray!25}{white!0}
	\begin{longtable}{@{}|>{\centering\arraybackslash}m{.25\textwidth}|m{.5\textwidth}|>{\centering\arraybackslash}m{.1\textwidth}|@{}}
		
		Nome caso d'uso & Edit project \\
		Descrizione & \\
		Attori & \\
		Pre-condizioni & \\
		Sequenza delle azioni & \\
		Post-condizioni & \\
		Scenario alternativo & \\
		
	\end{longtable}
\end{table}

\begin{table}[H]
	\centering
	\caption{Use Case: Remove project}
	\label{tab:use-case-project-removal}
	\rowcolors{2}{gray!25}{white!0}
	\begin{longtable}{@{}|>{\centering\arraybackslash}m{.25\textwidth}|m{.5\textwidth}|>{\centering\arraybackslash}m{.1\textwidth}|@{}}
		
		Nome caso d'uso & Remove project \\
		Descrizione & \\
		Attori & \\
		Pre-condizioni & \\
		Sequenza delle azioni & \\
		Post-condizioni & \\
		Scenario alternativo & \\
		
	\end{longtable}
\end{table}

\begin{table}[H]
	\centering
	\caption{Use Case: Share project}
	\label{tab:use-case-sharing-project}
	\rowcolors{2}{gray!25}{white!0}
	\begin{longtable}{@{}|>{\centering\arraybackslash}m{.25\textwidth}|m{.5\textwidth}|>{\centering\arraybackslash}m{.1\textwidth}|@{}}
		
		Nome caso d'uso & Share project \\
		Descrizione & \\
		Attori & \\
		Pre-condizioni & \\
		Sequenza delle azioni & \\
		Post-condizioni & \\
		Scenario alternativo & \\
		
	\end{longtable}
\end{table}

\begin{table}[H]
	\centering
	\caption{Use Case: Upload video}
	\label{tab:use-case-video-upload}
	\rowcolors{2}{gray!25}{white!0}
	\begin{longtable}{@{}|>{\centering\arraybackslash}m{.25\textwidth}|m{.5\textwidth}|>{\centering\arraybackslash}m{.1\textwidth}|@{}}
		
		Nome caso d'uso & Upload video\\
		Descrizione & \\
		Attori & \\
		Pre-condizioni & \\
		Sequenza delle azioni & \\
		Post-condizioni & \\
		Scenario alternativo & \\
		
	\end{longtable}
\end{table}

\begin{table}[H]
	\centering
	\caption{Use Case: Edit video}
	\label{tab:use-case-video-editing}
	\rowcolors{2}{gray!25}{white!0}
	\begin{longtable}{@{}|>{\centering\arraybackslash}m{.25\textwidth}|m{.5\textwidth}|>{\centering\arraybackslash}m{.1\textwidth}|@{}}
		
		Nome caso d'uso & Edit video\\
		Descrizione & \\
		Attori & \\
		Pre-condizioni & \\
		Sequenza delle azioni & \\
		Post-condizioni & \\
		Scenario alternativo & \\
		
	\end{longtable}
\end{table}

\begin{table}[H]
	\centering
	\caption{Use Case: Remove video}
	\label{tab:use-case-video-removal}
	\rowcolors{2}{gray!25}{white!0}
	\begin{longtable}{@{}|>{\centering\arraybackslash}m{.25\textwidth}|m{.5\textwidth}|>{\centering\arraybackslash}m{.1\textwidth}|@{}}
		
		Nome caso d'uso & Remove video\\
		Descrizione & \\
		Attori & \\
		Pre-condizioni & \\
		Sequenza delle azioni & \\
		Post-condizioni & \\
		Scenario alternativo & \\
		
	\end{longtable}
\end{table}

\begin{table}[H]
	\centering
	\caption{Use Case: view report}
	\label{tab:use-case-report-view}
	\rowcolors{2}{gray!25}{white!0}
	\begin{longtable}{@{}|>{\centering\arraybackslash}m{.25\textwidth}|m{.5\textwidth}|>{\centering\arraybackslash}m{.1\textwidth}|@{}}
		
		Nome caso d'uso & View report \\
		Descrizione & \\
		Attori & \\
		Pre-condizioni & \\
		Sequenza delle azioni & \\
		Post-condizioni & \\
		Scenario alternativo & \\
		
	\end{longtable}
\end{table}

\begin{table}[H]
	\centering
	\caption{Use Case: View video report}
	\label{tab:use-case-video-report-view}
	\rowcolors{2}{gray!25}{white!0}
	\begin{longtable}{@{}|>{\centering\arraybackslash}m{.25\textwidth}|m{.5\textwidth}|>{\centering\arraybackslash}m{.1\textwidth}|@{}}
		
		Nome caso d'uso & View video report \\
		Descrizione & \\
		Attori & \\
		Pre-condizioni & \\
		Sequenza delle azioni & \\
		Post-condizioni & \\
		Scenario alternativo & \\
		
	\end{longtable}
\end{table}

\begin{table}[H]
	\centering
	\caption{Use Case: View project report}
	\label{tab:use-case-project-report-view}
	\rowcolors{2}{gray!25}{white!0}
	\begin{longtable}{@{}|>{\centering\arraybackslash}m{.25\textwidth}|m{.5\textwidth}|>{\centering\arraybackslash}m{.1\textwidth}|@{}}
		
		Nome caso d'uso & View project report\\
		Descrizione & \\
		Attori & \\
		Pre-condizioni & \\
		Sequenza delle azioni & \\
		Post-condizioni & \\
		Scenario alternativo & \\
		
	\end{longtable}
\end{table}

\begin{table}[H]
	\centering
	\caption{Use Case: Download report}
	\label{tab:use-case-report-download}
	\rowcolors{2}{gray!25}{white!0}
	\begin{longtable}{@{}|>{\centering\arraybackslash}m{.25\textwidth}|m{.5\textwidth}|>{\centering\arraybackslash}m{.1\textwidth}|@{}}
		
		Nome caso d'uso & Download report\\
		Descrizione & \\
		Attori & \\
		Pre-condizioni & \\
		Sequenza delle azioni & \\
		Post-condizioni & \\
		Scenario alternativo & \\
		
	\end{longtable}
\end{table}

\begin{table}[H]
	\centering
	\caption{Use Case: Update user data}
	\label{tab:use-case-user-data-update}
	\rowcolors{2}{gray!25}{white!0}
	\begin{longtable}{@{}|>{\centering\arraybackslash}m{.25\textwidth}|m{.5\textwidth}|>{\centering\arraybackslash}m{.1\textwidth}|@{}}
		Nome caso d'uso & Update user data \\
		Descrizione & \\
		Attori & \\
		Pre-condizioni & \\
		Sequenza delle azioni & \\
		Post-condizioni & \\
		Scenario alternativo & \\

	\end{longtable}
\end{table}

\begin{table}[H]
	\centering
	\caption{Use Case: Delete account}
	\label{tab:use-case-delete-account}
	\rowcolors{2}{gray!25}{white!0}
	\begin{longtable}{@{}|>{\centering\arraybackslash}m{.25\textwidth}|m{.5\textwidth}}
		Nome caso d'uso & Delete account \\
		Descrizione & \\
		Attori & \\
		Pre-condizioni & \\
		Sequenza delle azioni & \\
		Post-condizioni & \\
		Scenario alternativo & 
	\end{longtable}
\end{table}
\newpage
\subsection{Modello dei dati}
Di seguito è riportato il diagramma entità relazioni concettuale:
% Copyright (c)  2019  FSC.
% Permission is granted to copy, distribute and/or modify this document
% under the terms of the GNU Free Documentation License, Version 1.3
% or any later version published by the Free Software Foundation;
% with no Invariant Sections, no Front-Cover Texts, and no Back-Cover Texts.
% A copy of the license is included in the section entitled "GNU
% Free Documentation License".

\begin{tikzpicture}
%% Entities %%
\erentity{Users}{%
	\erkey{id}\\
	name\\
	surname\\
	email\\
	password
};

\erentity[right=5cm of Users]{Projects}{%
	\erkey{id}\\
	name
};

\erentity[below=4cm of Projects]{Videos}{%
	\erkey{id}\\
	name\\
	framerate\\
	start\\
	end\\
	duration\\
	url\\
	report
};

%% Relations %%
\errelation[right=2.8cm of Users]{creator};
\errelation[below=6.48cm of Users]{author};
\errelation[below right=2.5cm and 6.94cm of Projects]{contains};
\errelation[right=11cm of Projects]{holds};
\errelation[above right= 1.65cm and 3cm of Users]{permissions};

%% Associations %%
%Projects - holds - Projects
\erassoc[->]{Projects}{holds}{(0,N)}{above};
\node[inner sep=0,above=2cm of Projects] (help1) at (Projects) {};
\node[inner sep=0,above=2cm of holds] (help2) at (holds) {};
\erassoc[->]{help1}{Projects}{(0,1)}{right};
\erassoc{help1}{help2}{}{};
\erassoc{holds}{help2}{}{};
%Users - creator - Projects
\erassoc{Users}{creator}{(0,N)}{above};
\erassoc{Projects}{creator}{(1,1)}{above};
%Users - author - Videos
\erassoc{Users}{author}{(0,N)}{right};
\erassoc{Videos}{author}{(1,1)}{above};
%Projects - contains - Videos
\erassoc{Projects}{contains}{(0,N)}{right};
\erassoc{Videos}{contains}{(1,1)}{right};
%Users - permissions - Projects
\node[inner sep=0,above=2cm of Users] (help3) at (Users) {};
\node[inner sep=0,above left=2cm and 0.5cm of permissions] (help4) at 
(Projects) {};
\node[inner sep=0,above left=0.7cm and 0.5cm of permissions] (help5) at 
(Projects) {};
\erassoc{help3}{permissions}{(0,N)}{above left};
\erassoc{help4}{permissions}{(1,N)}{above right};
\erassoc{help3}{Users}{}{};
\erassoc{help4}{help5}{}{};

%% permissions attributes %%
\erattr{above}{read?}{2,2.5}{attr1};
\erattr{above}{modify?}{3.2,3}{attr2};
\erattr{above}{add?}{4.4,3}{attr3};
\erattr{above}{remove?}{5.4,2.5}{attr4};
\erassoc{permissions}{attr1}{}{right};
\erassoc{permissions}{attr2}{}{right};
\erassoc{permissions}{attr3}{}{right};
\erassoc{permissions}{attr4}{}{right};
\end{tikzpicture}

\subsection{Sicurezza e privacy}
Per quanto riguarda i sistemi di sicurezza, la piattaforma web è suddivisa in 
due macrosezioni: l'area pubblica (landing page) e l'area privata.

Nella landing page, ogni visitatore può accedervi liberamente con lo scopo di 
visionare quello che la web app offre prima di effettuare la registrazione. In 
questa area pubblica, inoltre, è possibile effettuare dei test delle emozioni 
senza, per quanto riguarda il team e la finalità di questo progetto, salvare 
dati sensibili.

Per quanto riguarda l'area privata, l'accesso è consentito solo attraverso il 
login. Da qui l'utente loggato avrà a disposizione di tutti gli strumenti utili 
per poter effettuare un'analisi. L'utente, inoltre, per poter accedere dovrà 
quindi effettuare una prima fase di registrazione dove dovrà fornire dati 
sensibili quali email, password, nome, cognome e sesso.

\section{Requisiti di contenuto}\label{sec:requisiti-di-contenuto}
% TODO: Complete table of contents

\begin{table}[H]
	\centering
	\caption{Contenuti di Emotionally.}
	\label{tab:bisogni-utenti}
	\rowcolors{2}{gray!25}{white!0}
	\begin{longtable}{@{}|>{\centering\arraybackslash}m{.25\textwidth}|m{.25\textwidth}|m{.25\textwidth}|>{\centering\arraybackslash}m{.1\textwidth}|@{}}
		\hline
		\rowcolor{emotionally-color}
		{\color{white} \textbf{Sezione}}   & {\color{white} 
		\textbf{Sottosezione}}     & {\color{white} \textbf{Requisiti di 
		contenuto}} & {\color{white} \textbf{Dove trovare le informazioni}} 
		\\\hline
		\endfirsthead
		\cellcolor{white!0}  & Try me & Questa sezione ha lo scopo di far 
		effettuare un'analisi delle emozioni di prova agli utenti visitatori 
		della piattaforma web. & Non sono previste informazioni. \\
		\cellcolor{white!0}  & Funzionalità & Questa sezione ha lo scopo di 
		presentare 
		le funzionalità offerte da Emotionally attraverso una breve descrizione 
		delle stesse. & Non è possibile reperire le informazioni in quanto sono 
		state create dal team FSC.\\
		\multirow{-2}{*}{Landing page} & Su di noi & Questa sezione ha lo scopo 
		di 
		presentare i componenti del team sviluppatore di Emotionally. & In 
		quanto le informazioni riguardano i componenti del team, non è 
		possibile reperire tali informazioni. \\		
		\hline
	\end{longtable}
\end{table}

\section{Requisiti di gestione}\label{sec:requisiti-di-gestione}
La memorizzazzione dei dati sensibili dell'utente e dei suoi progetti verrà 
effettuata su una base di dati creata dal team FSC. Gli stessi verranno 
sottoposti a backup sistematici per evitare la loro perdita nel momento in cui 
il sistema si interrompa per qualsiasi causa.

Il team FSC, posto dal committente come gruppo produttore della piattaforma 
web, si occuperà della manutenzione ordinaria e straordinaria del sistema, 
affinchè gli obiettivi definiti di comune accordo non cessino di funzionare.

Per quanto riguarda la gestione dei contenuti di Emotionally, saranno gli 
utenti stessi che inseriranno i contenuti che saranno i video da analizzare per 
le proprie analisi senza doversi preoccupare della struttura architetturale 
implementata, permettendo la semplicità nelle operazioni di upload. Il sistema 
all'analisi di un progetto o di un singolo video renderà disponibile un report 
che l'utente può visionare e, se vuole, scaricare. 
Sarà compito del team FSC a redigere i contenuti della landing page, dove 
verranno riportate descrizioni delle funzionalità del sistema, concordati con 
il committente.

\section{Requisiti di accessibilità}\label{sec:requisiti-di-accessibilita}
% TODO: Insert accessibility requirements

\section{Requisiti di usabilità}\label{sec:requisiti-di-usabilita}
% TODO: Insert usability requirements