% Copyright (c)  2019  FSC.
% Permission is granted to copy, distribute and/or modify this document
% under the terms of the GNU Free Documentation License, Version 1.3
% or any later version published by the Free Software Foundation;
% with no Invariant Sections, no Front-Cover Texts, and no Back-Cover Texts.
% A copy of the license is included in the section entitled "GNU
% Free Documentation License".

\chapter{Requisiti del sito}\label{chap:requisiti-del-sito}

\section{Requisiti di architettura}\label{sec:requisiti-di-architettura}

L'architettura informativa illustrata successivamente definisce una struttura 
generale dello \textit{schema gerarchico} indicante la logica di 
\textit{navigazione} suddividendo la web app in sezioni e sottosezioni:

% TODO: Insert an architectural structure.

Indicativamente, la struttura di navigazione delle pagine più importante è 
definita mediante le seguenti \textit{gabbie logiche grafiche minimali}. Si 
riserverà al web designer il compito di definire meccanismi di navigazione e 
layout grafici più avanzati che ne determineranno il risultato finale.

% TODO: Insert a logical cages.

\section{Requisiti di comunicazione}\label{sec:requisiti-di-comunicazione}
Le pagine interne della piattaforma web Emotionally dovranno essere 
graficamente coerenti tra di loro, fatta eccezione per la pagina di landing che 
dovrà presentare la piattaforma all'utente. La piattaforma web sarà bilingue 
(italiano e inglese).

Ogni pagina del sistema Emotionally dovrà:
\begin{itemize}
	\item contenere il logo del sistema stesso posizionato in alto a sinistra,
	\item avere un comportamento responsive: il sito deve ridimensionarsi 
	automaticamente al variare della risoluzionie video e della dimnesione 
	della finestra del browser,
	\item contenere caratteri la cui dimensione dovrà essere modificabile 
	dall'utente,
	\item avere una funzione accessibile, almeno di livello AA.
\end{itemize}
La landing page presenta, attraverso un'interfaccia chiara e usabile, le 
funzionalità del sistema e permette all'utente di approcciare il riconoscimento 
delle emozioni. 

Inoltre, per sottolineare la divisione tra il sistema e la landing page si è 
deciso di utilizzare un bottone di login differente rispetto allo stile dei 
link presenti nella navbar. A seguito del click del bottone, l'utente viene 
reindirizzato alla pagina di login che rappresenta, seppur in parte, lo stile 
grafico del sistema.

\section{Requisiti funzionali}\label{sec:requisiti-funzionali}

In questa sezione vengono illustrati i casi d'uso e le relative 
rappresentazioni tabellari riferiti alle diverse funzionalità che Emotionally 
offre ai suoi utenti. 

\subsection{Casi d'uso}
% TODO: Modify use cases
\begin{tikzpicture} 
\begin{umlsystem}[x=6, y=20]{Emotionally} 
% Define uses case of the GUEST
 	\umlusecase[y=-2, name=Registration]{Registration} 
 	\umlusecase[y=-3, name=TestAnalysis]{Test analysis}
% Define uses case of the USER
 	\umlusecase[y=-4, name=Login]{Login}
 	\umlusecase[y=-5, name=Logout]{Logout}
 	\umlusecase[y=-6, name=ProjectCreation]{Project creation}
 	\umlusecase[x=6, y=-5, name=ProjectEditing]{Project editing}
 	\umlusecase[x=6, y=-6, name=ProjectRemoval]{Project removal}
 	\umlusecase[y=-7, name=ProjectView]{Project view}
 	\umlusecase[x=6, y=-7, name=VideoUpload]{Video upload}
 	\umlusecase[x=6, y=-8, name=VideoEditing]{Video editing}
 	\umlusecase[x=6, y=-9, name=VideoRemoval]{Video removal}
 	\umlusecase[x=6, y=-10, name=SharingProject]{Sharing project}
 	\umlusecase[y=-8, name=ReportView]{Report view}
 	\umlusecase[x=6, y=-7, name=VideoReportView]{Video report view}
 	\umlusecase[x=6, y=-8, name=ProjectReportView]{Project report view}
 	\umlusecase[x=6, y=-9, name=ReportDownload]{Report download}
 	\umlusecase[y=-9, name=UserDataUpdate]{User data update}
 	\umlusecase[y=-10, name=DeleteAccount]{Delete account}
\end{umlsystem} 

% Define actor
\umlactor[x=1, y=17.5]{Guest}
\umlactor[x=1, y=14.5]{User}
\umlactor[y=10.5]{Designer}
\umlactor[x=2, y=10.5]{Analyst}

% Define generalizzation between actor
\umlinherit{User}{Guest}
\umlinherit{Designer}{User}
\umlinherit{Analyst}{User}

% Define associations between actors and use cases
\umlassoc{Guest}{Registration} 
\umlassoc{Guest}{TestAnalysis} 
\umlassoc{User}{Login} 
\umlassoc{User}{Logout} 
\umlassoc{User}{ProjectCreation}
\umlassoc{User}{ProjectView}
\umlassoc{User}{ReportView}
\umlassoc{User}{UserDataUpdate}
\umlassoc{User}{DeleteAccount}

%Define extensions between use cases
\umlVHextend{ProjectEditing}{ProjectView}
\umlVHextend{ProjectRemoval}{ProjectView} 
\umlVHextend{SharingProject}{ProjectView} 
\umlVHextend{VideoUpload}{ProjectView} 
\umlVHextend{VideoEditing}{ProjectView} 
\umlVHextend{VideoRemoval}{ProjectView}  
\umlVHextend{ReportDownload}{ReportView}
 
% Define generalizzation between use cases
\umlinherit{VideoReportView}{ReportView}
\umlinherit{ProjectReportView}{ReportView}

\end{tikzpicture}

\subsection{Rappresentazione tabellare}
% TODO: Complete use case scenarios
\begin{table}[H]
	\centering
	\caption{Use Case: Test analysis}
	\label{tab:use-case-test-analysis}
	\rowcolors{2}{gray!25}{white!0}
	\begin{longtable}{@{}|>{\centering\arraybackslash}m{.25\textwidth}|m{.5\textwidth}|>{\centering\arraybackslash}m{.1\textwidth}|@{}}
		
		Nome caso d'uso & Registration \\
		Descrizione & \\
		Attori & \\
		Pre-condizioni & \\
		Sequenza delle azioni & \\
		Post-condizioni & \\
		Scenario alternativo & \\
		
	\end{longtable}
\end{table}

\begin{table}[H]
	\centering
	\caption{Use Case: Registration}
	\label{tab:use-case-registration}
	\rowcolors{2}{gray!25}{white!0}
	\begin{longtable}{@{}|>{\centering\arraybackslash}m{.25\textwidth}|m{.5\textwidth}|>{\centering\arraybackslash}m{.1\textwidth}|@{}}
		
		Nome caso d'uso & Test analysis \\
		Descrizione & \\
		Attori & \\
		Pre-condizioni & \\
		Sequenza delle azioni & \\
		Post-condizioni & \\
		Scenario alternativo & \\
		
	\end{longtable}
\end{table}

\begin{table}[H]
	\centering
	\caption{Use Case: Login}
	\label{tab:use-case-login}
	\rowcolors{2}{gray!25}{white!0}
	\begin{longtable}{@{}|>{\centering\arraybackslash}m{.25\textwidth}|m{.5\textwidth}|>{\centering\arraybackslash}m{.1\textwidth}|@{}}
		
		Nome caso d'uso & Login \\
		Descrizione & \\
		Attori & \\
		Pre-condizioni & \\
		Sequenza delle azioni & \\
		Post-condizioni & \\
		Scenario alternativo & \\
		
	\end{longtable}
\end{table}

\begin{table}[H]
	\centering
	\caption{Use Case: Logout}
	\label{tab:use-case-logout}
	\rowcolors{2}{gray!25}{white!0}
	\begin{longtable}{@{}|>{\centering\arraybackslash}m{.25\textwidth}|m{.5\textwidth}|>{\centering\arraybackslash}m{.1\textwidth}|@{}}
		
		Nome caso d'uso & Logout \\
		Descrizione & \\
		Attori & \\
		Pre-condizioni & \\
		Sequenza delle azioni & \\
		Post-condizioni & \\
		Scenario alternativo & \\
		
	\end{longtable}
\end{table}

\begin{table}[H]
	\centering
	\caption{Use Case: Project creation}
	\label{tab:use-case-project-creation}
	\rowcolors{2}{gray!25}{white!0}
	\begin{longtable}{@{}|>{\centering\arraybackslash}m{.25\textwidth}|m{.5\textwidth}|>{\centering\arraybackslash}m{.1\textwidth}|@{}}
		
		Nome caso d'uso & Project Creation \\
		Descrizione & \\
		Attori & \\
		Pre-condizioni & \\
		Sequenza delle azioni & \\
		Post-condizioni & \\
		Scenario alternativo & \\
		
	\end{longtable}
\end{table}

\begin{table}[H]
	\centering
	\caption{Use Case: Project view}
	\label{tab:use-case-project-view}
	\rowcolors{2}{gray!25}{white!0}
	\begin{longtable}{@{}|>{\centering\arraybackslash}m{.25\textwidth}|m{.5\textwidth}|>{\centering\arraybackslash}m{.1\textwidth}|@{}}
		
		Nome caso d'uso & Project view \\
		Descrizione & \\
		Attori & \\
		Pre-condizioni & \\
		Sequenza delle azioni & \\
		Post-condizioni & \\
		Scenario alternativo & \\
		
	\end{longtable}
\end{table}

\begin{table}[H]
	\centering
	\caption{Use Case: Project editing}
	\label{tab:use-case-project-editing}
	\rowcolors{2}{gray!25}{white!0}
	\begin{longtable}{@{}|>{\centering\arraybackslash}m{.25\textwidth}|m{.5\textwidth}|>{\centering\arraybackslash}m{.1\textwidth}|@{}}
		
		Nome caso d'uso & Project editing \\
		Descrizione & \\
		Attori & \\
		Pre-condizioni & \\
		Sequenza delle azioni & \\
		Post-condizioni & \\
		Scenario alternativo & \\
		
	\end{longtable}
\end{table}

\begin{table}[H]
	\centering
	\caption{Use Case: Project removal}
	\label{tab:use-case-project-removal}
	\rowcolors{2}{gray!25}{white!0}
	\begin{longtable}{@{}|>{\centering\arraybackslash}m{.25\textwidth}|m{.5\textwidth}|>{\centering\arraybackslash}m{.1\textwidth}|@{}}
		
		Nome caso d'uso & Project removal \\
		Descrizione & \\
		Attori & \\
		Pre-condizioni & \\
		Sequenza delle azioni & \\
		Post-condizioni & \\
		Scenario alternativo & \\
		
	\end{longtable}
\end{table}

\begin{table}[H]
	\centering
	\caption{Use Case: Sharing project}
	\label{tab:use-case-sharing-project}
	\rowcolors{2}{gray!25}{white!0}
	\begin{longtable}{@{}|>{\centering\arraybackslash}m{.25\textwidth}|m{.5\textwidth}|>{\centering\arraybackslash}m{.1\textwidth}|@{}}
		
		Nome caso d'uso & Sharing project \\
		Descrizione & \\
		Attori & \\
		Pre-condizioni & \\
		Sequenza delle azioni & \\
		Post-condizioni & \\
		Scenario alternativo & \\
		
	\end{longtable}
\end{table}

\begin{table}[H]
	\centering
	\caption{Use Case: Video upload}
	\label{tab:use-case-video-upload}
	\rowcolors{2}{gray!25}{white!0}
	\begin{longtable}{@{}|>{\centering\arraybackslash}m{.25\textwidth}|m{.5\textwidth}|>{\centering\arraybackslash}m{.1\textwidth}|@{}}
		
		Nome caso d'uso & Video upload \\
		Descrizione & \\
		Attori & \\
		Pre-condizioni & \\
		Sequenza delle azioni & \\
		Post-condizioni & \\
		Scenario alternativo & \\
		
	\end{longtable}
\end{table}

\begin{table}[H]
	\centering
	\caption{Use Case: Video editing}
	\label{tab:use-case-video-editing}
	\rowcolors{2}{gray!25}{white!0}
	\begin{longtable}{@{}|>{\centering\arraybackslash}m{.25\textwidth}|m{.5\textwidth}|>{\centering\arraybackslash}m{.1\textwidth}|@{}}
		
		Nome caso d'uso & Video editing \\
		Descrizione & \\
		Attori & \\
		Pre-condizioni & \\
		Sequenza delle azioni & \\
		Post-condizioni & \\
		Scenario alternativo & \\
		
	\end{longtable}
\end{table}

\begin{table}[H]
	\centering
	\caption{Use Case: Video removal}
	\label{tab:use-case-video-removal}
	\rowcolors{2}{gray!25}{white!0}
	\begin{longtable}{@{}|>{\centering\arraybackslash}m{.25\textwidth}|m{.5\textwidth}|>{\centering\arraybackslash}m{.1\textwidth}|@{}}
		
		Nome caso d'uso & Registration \\
		Descrizione & \\
		Attori & \\
		Pre-condizioni & \\
		Sequenza delle azioni & \\
		Post-condizioni & \\
		Scenario alternativo & \\
		
	\end{longtable}
\end{table}

\begin{table}[H]
	\centering
	\caption{Use Case: Report view}
	\label{tab:use-case-report-view}
	\rowcolors{2}{gray!25}{white!0}
	\begin{longtable}{@{}|>{\centering\arraybackslash}m{.25\textwidth}|m{.5\textwidth}|>{\centering\arraybackslash}m{.1\textwidth}|@{}}
		
		Nome caso d'uso & Report view \\
		Descrizione & \\
		Attori & \\
		Pre-condizioni & \\
		Sequenza delle azioni & \\
		Post-condizioni & \\
		Scenario alternativo & \\
		
	\end{longtable}
\end{table}

\begin{table}[H]
	\centering
	\caption{Use Case: Video report view}
	\label{tab:use-case-video-report-view}
	\rowcolors{2}{gray!25}{white!0}
	\begin{longtable}{@{}|>{\centering\arraybackslash}m{.25\textwidth}|m{.5\textwidth}|>{\centering\arraybackslash}m{.1\textwidth}|@{}}
		
		Nome caso d'uso & Video report view \\
		Descrizione & \\
		Attori & \\
		Pre-condizioni & \\
		Sequenza delle azioni & \\
		Post-condizioni & \\
		Scenario alternativo & \\
		
	\end{longtable}
\end{table}

\begin{table}[H]
	\centering
	\caption{Use Case: Project report view}
	\label{tab:use-case-project-report-view}
	\rowcolors{2}{gray!25}{white!0}
	\begin{longtable}{@{}|>{\centering\arraybackslash}m{.25\textwidth}|m{.5\textwidth}|>{\centering\arraybackslash}m{.1\textwidth}|@{}}
		
		Nome caso d'uso & Project report view \\
		Descrizione & \\
		Attori & \\
		Pre-condizioni & \\
		Sequenza delle azioni & \\
		Post-condizioni & \\
		Scenario alternativo & \\
		
	\end{longtable}
\end{table}

\begin{table}[H]
	\centering
	\caption{Use Case: Report download}
	\label{tab:use-case-report-download}
	\rowcolors{2}{gray!25}{white!0}
	\begin{longtable}{@{}|>{\centering\arraybackslash}m{.25\textwidth}|m{.5\textwidth}|>{\centering\arraybackslash}m{.1\textwidth}|@{}}
		
		Nome caso d'uso & Report download \\
		Descrizione & \\
		Attori & \\
		Pre-condizioni & \\
		Sequenza delle azioni & \\
		Post-condizioni & \\
		Scenario alternativo & \\
		
	\end{longtable}
\end{table}

\begin{table}[H]
	\centering
	\caption{Use Case: User data update}
	\label{tab:use-case-user-data-update}
	\rowcolors{2}{gray!25}{white!0}
	\begin{longtable}{@{}|>{\centering\arraybackslash}m{.25\textwidth}|m{.5\textwidth}|>{\centering\arraybackslash}m{.1\textwidth}|@{}}
		Nome caso d'uso & User data update \\
		Descrizione & \\
		Attori & \\
		Pre-condizioni & \\
		Sequenza delle azioni & \\
		Post-condizioni & \\
		Scenario alternativo & \\

	\end{longtable}
\end{table}

\begin{table}[H]
	\centering
	\caption{Use Case: Delete account}
	\label{tab:use-case-delete-account}
	\rowcolors{2}{gray!25}{white!0}
	\begin{longtable}{@{}|>{\centering\arraybackslash}m{.25\textwidth}|m{.5\textwidth}}
		Nome caso d'uso & Delete account \\
		Descrizione & \\
		Attori & \\
		Pre-condizioni & \\
		Sequenza delle azioni & \\
		Post-condizioni & \\
		Scenario alternativo & 
	\end{longtable}
\end{table}

\subsection{Modello dei dati}
% TODO: Insert Diagram E-R

\subsection{Sicurezza e privacy}
Per quanto riguarda i sistemi di sicurezza, la piattaforma web è suddivisa in 
due macrosezioni: l'area pubblica (landing page) e l'area privata.

Nella landing page, ogni visitatore può accedervi liberamente con lo scopo di 
visionare quello che la web app offre prima di effettuare la registrazione. In 
questa area pubblica, inoltre, è possibile effettuare dei test delle emozioni 
senza, per quanto riguarda il team e la finalità di questo progetto, salvare 
dati sensibili.

Per quanto riguarda l'area privata, l'accesso è consentito solo attraverso il 
login. Da qui l'utente loggato avrà a disposizione di tutti gli strumenti utili 
per poter effettuare un'analisi. L'utente, inoltre, per poter accedere dovrà 
quindi effettuare una prima fase di registrazione dove dovrà fornire dati 
sensibili quali email, password, nome, cognome e sesso.

\section{Requisiti di contenuto}\label{sec:requisiti-di-contenuto}
% TODO: Complete table of contents

\begin{table}[H]
	\centering
	\caption{I bisogni degli utenti di Emotionally.}
	\label{tab:bisogni-utenti}
	\rowcolors{2}{gray!25}{white!0}
	\begin{longtable}{@{}|>{\centering\arraybackslash}m{.25\textwidth}|m{.25\textwidth}|m{.25\textwidth}|>{\centering\arraybackslash}m{.1\textwidth}|@{}}
		\hline
		\rowcolor{emotionally-color}
		{\color{white} \textbf{Sezione}}   & {\color{white} 
		\textbf{Sottosezione}}     & {\color{white} \textbf{Requisiti di 
		contenuto}} & {\color{white} \textbf{Dove trovare le informazioni}} 
		\\\hline
		\endfirsthead
		\hline
	\end{longtable}
\end{table}
\section{Requisiti di gestione}\label{sec:requisiti-di-gestione}

\section{Requisiti di accessibilità}\label{sec:requisiti-di-accessibilita}

\section{Requisiti di usabilità}\label{sec:requisiti-di-usabilita}
