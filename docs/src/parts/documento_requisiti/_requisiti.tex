% Copyright (c)  2019  FSC.
% Permission is granted to copy, distribute and/or modify this document
% under the terms of the GNU Free Documentation License, Version 1.3
% or any later version published by the Free Software Foundation;
% with no Invariant Sections, no Front-Cover Texts, and no Back-Cover Texts.
% A copy of the license is included in the section entitled "GNU
% Free Documentation License".

\chapter{Requisiti del sito}\label{chap:requisiti-del-sito}

\section{Requisiti di architettura}\label{sec:requisiti-di-architettura}

L'architettura informativa illustrata successivamente definisce una struttura 
generale dello \textit{schema gerarchico} indicante la logica di 
\textit{navigazione} suddividendo la web app in sezioni e sottosezioni:

% TODO: Insert an architectural structure.

Indicativamente, la struttura di navigazione delle pagine più importante è 
definita mediante le seguenti \textit{gabbie logiche grafiche minimali}. Si 
riserverà al web designer il compito di definire meccanismi di navigazione e 
layout grafici più avanzati che ne determineranno il risultato finale.

% TODO: Insert a logical cages.

\section{Requisiti di comunicazione}\label{sec:requisiti-di-comunicazione}
Le pagine interne della piattaforma web Emotionally dovranno essere 
graficamente coerenti tra di loro, fatta eccezione per la pagina di landing che 
dovrà presentare la piattaforma all'utente. La piattaforma web sarà bilingue 
(italiano e inglese).

Ogni pagina del sistema Emotionally dovrà:
\begin{itemize}
	\item contenere il logo del sistema stesso posizionato in alto a sinistra,
	\item avere un comportamento responsive: il sito deve ridimensionarsi 
	automaticamente al variare della risoluzionie video e della dimnesione 
	della finestra del browser,
	\item contenere caratteri la cui dimensione dovrà essere modificabile 
	dall'utente,
	\item avere una funzione accessibile, almeno di livello AA.
\end{itemize}
La landing page presenta, attraverso un'interfaccia chiara e usabile, le 
funzionalità del sistema e permette all'utente di approcciare il riconoscimento 
delle emozioni. 

Inoltre, per sottolineare la divisione tra il sistema e la landing page si è 
deciso di utilizzare un bottone di login differente rispetto allo stile dei 
link presenti nella navbar. A seguito del click del bottone, l'utente viene 
reindirizzato alla pagina di login che rappresenta, seppur in parte, lo stile 
grafico del sistema.

\section{Requisiti funzionali}\label{sec:requisiti-funzionali}

In questa sezione vengono illustrati i casi d'uso e le relative 
rappresentazioni tabellari riferiti alle diverse funzionalità che Emotionally 
offre ai suoi utenti. 

\subsection{Casi d'uso}
% TODO: Modify use cases
\begin{tikzpicture} 
\begin{umlsystem}[x=6, y=20]{Emotionally} 
% Define uses case of the GUEST
 	\umlusecase[y=-2, name=Registration]{Registration} 
 	\umlusecase[y=-3, name=TestAnalysis]{Test analysis}
% Define uses case of the USER
 	\umlusecase[y=-4, name=Login]{Login}
 	\umlusecase[y=-5, name=Logout]{Logout}
 	\umlusecase[y=-6, name=ProjectCreation]{Project creation}
 	\umlusecase[x=6, y=-5, name=ProjectEditing]{Project editing}
 	\umlusecase[x=6, y=-6, name=ProjectRemoval]{Project removal}
 	\umlusecase[y=-7, name=ProjectView]{Project view}
 	\umlusecase[x=6, y=-7, name=VideoUpload]{Video upload}
 	\umlusecase[x=6, y=-8, name=VideoEditing]{Video editing}
 	\umlusecase[x=6, y=-9, name=VideoRemoval]{Video removal}
 	\umlusecase[x=6, y=-10, name=SharingProject]{Sharing project}
 	\umlusecase[y=-8, name=ReportView]{Report view}
 	\umlusecase[x=6, y=-7, name=VideoReportView]{Video report view}
 	\umlusecase[x=6, y=-8, name=ProjectReportView]{Project report view}
 	\umlusecase[x=6, y=-9, name=ReportDownload]{Report download}
 	\umlusecase[y=-9, name=UserDataUpdate]{User data update}
 	\umlusecase[y=-10, name=DeleteAccount]{Delete account}
\end{umlsystem} 

% Define actor
\umlactor[x=1, y=17.5]{Guest}
\umlactor[x=1, y=14.5]{User}
\umlactor[y=10.5]{Designer}
\umlactor[x=2, y=10.5]{Analyst}

% Define generalizzation between actor
\umlinherit{User}{Guest}
\umlinherit{Designer}{User}
\umlinherit{Analyst}{User}

% Define associations between actors and use cases
\umlassoc{Guest}{Registration} 
\umlassoc{Guest}{TestAnalysis} 
\umlassoc{User}{Login} 
\umlassoc{User}{Logout} 
\umlassoc{User}{ProjectCreation}
\umlassoc{User}{ProjectView}
\umlassoc{User}{ReportView}
\umlassoc{User}{UserDataUpdate}
\umlassoc{User}{DeleteAccount}

%Define extensions between use cases
\umlVHextend{ProjectEditing}{ProjectView}
\umlVHextend{ProjectRemoval}{ProjectView} 
\umlVHextend{SharingProject}{ProjectView} 
\umlVHextend{VideoUpload}{ProjectView} 
\umlVHextend{VideoEditing}{ProjectView} 
\umlVHextend{VideoRemoval}{ProjectView}  
\umlVHextend{ReportDownload}{ReportView}
 
% Define generalizzation between use cases
\umlinherit{VideoReportView}{ReportView}
\umlinherit{ProjectReportView}{ReportView}

\end{tikzpicture}

\subsection{Rappresentazione tabellare}

\section{Requisiti di contenuto}\label{sec:requisiti-di-contenuto}

\section{Requisiti di gestione}\label{sec:requisiti-di-gestione}

\section{Requisiti di accessibilità}\label{sec:requisiti-di-accessibilita}

\section{Requisiti di usabilità}\label{sec:requisiti-di-usabilita}
